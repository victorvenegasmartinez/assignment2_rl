\item \points{4b}

In this question, we'll train the agent with DeepMind's architecture on the Atari \texttt{Pong} environment. Run the following command to start the training process:
\begin{lstlisting}
$ python run.py --config_filename=q4_dqn
\end{lstlisting}
To speed up training, we have trained the model for 5 million steps (these pretrained weights will be automatically loaded once you run the above command). You are responsible for training it to completion, which should take \textbf{6 hours}. You should get a score of around 12-15 after 4 million total time steps.  As stated previously, the DeepMind paper claims average human performance is $ -3 $. Once your model has fully trained download the following file to your local machine ~src/submission/model.weights~ and include these weights with your code submission to Gradescope. 

\textit{Note: the weights file needs to be in the submission folder for the autograder to read them when you run the grader on your local machine.}


As the training time is roughly 6 hours, you may want to check after a few epochs that your network is making progress.  The following are some training tips:

\begin{itemize}
\item If you terminate your terminal session, any training processes which are running within this session will terminate.  In order to avoid this, you can start a session with Tmux which will persist even if you lose your connection to the vm. See the Azure guide for further details).
\item The evaluation score printed on terminal should start at 6 and increase.
\item The max of the q values should also be increasing.
\item The standard deviation of q shouldn't be too small. Otherwise it means that all states have similar q values.
\item Please find our Tensorboard graphs from one training session below.
\end{itemize}

\begin{figure}[H]
\centering
  \includegraphics[width=.4\linewidth]{images/Eval_R.png}
  \includegraphics[width=.4\linewidth]{images/Max_R.png}
  \includegraphics[width=.4\linewidth]{images/Avg_R.png}
  \includegraphics[width=.4\linewidth]{images/Max_Q.png}
\end{figure}